\documentclass{../vespers-booklet}
\usepackage{multicol}

\title{In Festo Omnium Sanctorum\\Ad I. Vesperas}
\author{First Vespers of the Feast of All Saints}

\begin{document}

\chapter*{Dominica Ad Vesperas}

%\section*{Vespers on Sunday}

\textit{The following prayers are said silently before the Office:}

\begin{latinenglishsection}

\latinenglish{
	Aperi, + Dómine, os meum ad benedicéndum nomen sanctum tuum:
	+ munda quoque cor meum ab ómnibus vanis, pervérsis et aliénis cogitatiónibus;
	intelléctum illúmina, afféctum inflámma, ut digne, atténte ac devóte hoc Offícium recitáre váleam,
	et exaudíri mérear ante conspéctum divínæ Majestátis tuæ.
	Per Christum Dóminum nostrum. 
	Amen.
	
	Domine, in unióne illíus divínæ intentiónis, qua ipse in terris laudes Deo persolvísti, hanc tibi Horam persólvo.
	
	Pater. Ave.
}{
	Open, + O Lord, my mouth to bless Thy holy Name; + cleanse also my heart from all vain, evil, and wandering thoughts; enlighten my understanding and kindle my affections; that I may worthily, attentively, and devoutly say this Office, and so merit to be heard before the presence of Thy divine Majesty.  Through Christ our Lord.  Amen.
	
	O Lord, in union with that divine intention wherewith Thou, whilst here on earth, didst render praises unto God, I desire to offer this Hour of prayer unto Thee.
}

 \end{latinenglishsection}

%\vfill

%\pagebreak
 
 \gresetinitiallines{1}
\gregorioscore{../common/deus-in-adjutorium}

\textit{
O God, come to my assistance.
\Vbar.~O Lord, make haste to help me.
Glory be to the Father, and to the Son, and to the Holy Spirit,
as it was in the beginning, is now, and ever shall be, world without end. Amen.
Alleluia.}

\section*{Psalm 109}

\textit{\textnormal{Ps.} The Lord said to my Lord: Sit thou at my right hand.}

\gresetinitiallines{1}
\gregorioscore{ps109-intonation}

%\input{../psalms/ps109-7}

 \begin{latinenglishsection}

\latinenglish{
	2. Donec ponam ini\textbf{mí}cos \textbf{tu}os,~*
		scabéllum \textbf{pe}dum tu\textbf{ó}rum.
	
	3. Virgam virtútis tuæ emíttet Dómi\textbf{nus} ex \textbf{Si}on:~*
		domináre in médio inimi\textbf{có}rum tu\textbf{ó}rum.
	
	4. Tecum princípium in die virtútis tuæ in splendóri\textbf{bus} sanc\textbf{tó}rum:~*
		ex útero ante lucíferum \textbf{gé}nu\textbf{i} te.
	
	5. Jurávit Dóminus, et non p{\oe}ni\textbf{té}bit \textbf{e}um:~*
		Tu es sacérdos in ætérnum secúndum órdi\textbf{nem} Mel\textbf{chí}sedech.
	
	6. Dóminus a \textbf{dex}tris \textbf{tu}is,~*
		confrégit in die iræ \textbf{su}æ \textbf{re}ges.
	
	7. Judicábit in natiónibus, im\textbf{plé}bit ru\textbf{í}nas:~*
		conquassábit cápita in \textbf{ter}ra mul\textbf{tó}rum.
	
	8. De torrénte in \textbf{vi}a \textbf{bi}bet:~*
		proptérea exal\textbf{tá}bit \textbf{ca}put.
	
	Glória \textbf{Pa}tri, et \textbf{Fí}lio,~*
		et Spi\textbf{rí}tui \textbf{Sanc}to.
}{
	\input{../psalms/english/ps109}
}

\end{latinenglishsection}

Sicut erat in princípio, et \textbf{nunc}, et \textbf{sem}per,~*
	et in s\'{\ae}cula sæcu\textbf{ló}rum. \textbf{A}men.
	
\gresetinitiallines{0}
\gregorioscore{ps109-antiphon}

\textit{\textnormal{Ant.} The Lord said to my Lord: Sit thou at my right hand.}


\end{document}